\pagestyle{plain}
\chapter{Introdução}

\section{Gestão de Ativos Intangíveis e Inovação Social}

A gestão estratégica de ativos intangíveis emerge como diferencial competitivo na economia do conhecimento, especialmente quando aplicada à valoração e proteção de saberes tradicionais. A fundamentação teórica remonta à teoria do crescimento da firma de Penrose \cite{Penrose1959}, que identificou recursos intangíveis e capacidades gerenciais como limitadores primários do crescimento organizacional. 

Na perspectiva da Resource-Based View \cite{Barney1991}, a vantagem competitiva sustentável deriva de recursos Valiosos, Raros, Inimitáveis e Não-substituíveis (VRIN), sendo o conhecimento tradicional um ativo que atende a esses critérios quando adequadamente protegido e gerido. Grant \cite{Grant1996} complementa essa visão ao posicionar conhecimento como recurso estratégico fundamental, propondo que a função primária da firma é integrar conhecimento especializado de indivíduos, conceito diretamente aplicável à gestão de saberes coletivos comunitários. 

A crise ambiental global aliada aos desafios da sustentabilidade revelou que modelos de gestão devem incorporar dimensões socioambientais e culturais, reconhecendo que o conhecimento tradicional representa ativo estratégico de alto valor \cite{Fletcher2021}. A Cúpula da Terra no Rio de Janeiro em 1992, por meio da Convenção para a Diversidade Biológica (CDB), estabeleceu marcos regulatórios que reconhecem o papel do conhecimento dos povos originários não apenas para a conservação da biodiversidade, mas como ativo intelectual passível de gestão, valoração e aplicação responsável \cite{Mekonen2017}.

Operacionalizar essa visão de conhecimentos tradicionais como ativos estratégicos VRIN demanda instrumentos jurídicos e mecanismos de proteção especializados. A propriedade intelectual (PI) constitui, neste contexto, instrumento jurídico central da economia do conhecimento, conferindo direitos exclusivos sobre criações intelectuais, invenções, obras artísticas, símbolos, nomes e desenhos comerciais, para incentivar inovação mediante garantia de apropriação de valor por criadores e investidores. No Brasil, a PI abrange propriedade industrial (\cite{brasil1996lei9279}: patentes, marcas, desenhos industriais), direitos autorais \cite{brasil1998lei9610}, programas de computador \cite{brasil1998lei9609}, cultivares \cite{brasil1997lei9456} e proteção de informações confidenciais (*know-how*, segredo industrial). Essa estrutura normativa, fundamentada em tratados internacionais tais como a convenção de Berna, TRIPS \cite{brasil1994decreto1355}, busca equilibrar proteção de direitos exclusivos com acesso social ao conhecimento.

No contexto de gestão da Propriedade Intelectual (PI), o conhecimento tradicional agroecológico configura-se como ativo intangível de relevância estratégica \cite{Mantyka-Pringle2017}. A teoria evolucionária de Nelson e Winter \cite{Nelson1982} fornece arcabouço conceitual para compreender conhecimento tradicional como conjunto de \textbf{rotinas organizacionais} comunitárias, padrões de comportamento recorrentes, tacitamente transmitidos e path-dependent, que estruturam respostas adaptativas a mudanças ambientais. 

A gestão eficaz desses ativos demanda o desenvolvimento de capacidades absortivas organizacionais \cite{CohenLevinthal1990}, que pode ser definida como a habilidade de reconhecer o valor de informação externa, assimilá-la e aplicá-la de forma sustentável. Zahra e George \cite{Zahra2002} refinaram esse conceito ao distinguir capacidade absortiva potencial (aquisição e assimilação) de capacidade absortiva realizada (transformação e exploração), distinção relevante para comunidades que buscam integrar tecnologias digitais (ML, plataformas) sem comprometer autenticidade cultural. Essa capacidade é construída cumulativamente (path-dependent) e constitui pré-requisito para que comunidades tradicionais e organizações parceiras possam apropriar-se efetivamente de conhecimentos e tecnologias externas, convertendo-os em inovação social e desenvolvimento territorial sustentável.

A interação entre conhecimentos tradicionais e práticas locais gera sistemas de inovação social que podem ser modelados, valorados e protegidos por meio de instrumentos de propriedade intelectual sui generis \cite{Olsson2001}. A gestão da inovação, conforme framework proposto por Tidd, Bessant e Pavitt \cite{Tidd2005}, envolve processos sistematizados de busca, seleção, implementação e captura de valor de oportunidades inovadoras, aplicável tanto a organizações quanto a comunidades que buscam salvaguardar e aplicar conhecimentos tradicionais. 

Von Hippel \cite{vonHippel1988} demonstrou que inovação frequentemente se origina de usuários (user innovation) que detêm conhecimento contextual especializado, conceito diretamente aplicável a comunidades tradicionais como fontes primárias de inovação agroecológica. O paradigma da \textbf{inovação aberta} \cite{Chesbrough2003} fornece arcabouço conceitual para integração estratégica de conhecimentos externos (outside-in), circulação de conhecimentos internos (inside-out) e co-criação com parceiros (coupled process), aplicável à gestão de conhecimentos tradicionais em ecossistemas de inovação territorial.

A gestão estratégica desses conhecimentos requer frameworks que integrem políticas de inovação, sistemas nacionais de inovação e marcos regulatórios apropriados \cite{Morin-Labatut1992}. O conhecimento tradicional, quando adequadamente protegido e valorado, pode gerar impacto social e contribuir para sustentabilidade através de estratégias de valoração de tecnologia, licenciamento responsável e parcerias colaborativas, caracterizando-se como sistema adaptativo complexo com alto potencial de inovação \cite{Mistry2016}.

A valoração de conhecimentos tradicionais como ativos intangíveis demanda metodologias sofisticadas que transcendam abordagens tradicionais baseadas exclusivamente em custos. A literatura sobre valoração de ativos intangíveis evoluiu de abordagens unidimensionais \cite{SmithParr2000} para frameworks analíticos que incorporam machine learning \cite{ZhouWang2022} e análise de contribuição tecnológica \cite{WuLi2022}, permitindo mensuração mais precisa do valor estratégico de conhecimentos coletivos. O modelo VAIC (Value Added Intellectual Coefficient) \cite{Pulic2000} oferece instrumento para mensurar eficiência na conversão de capital intelectual em valor, aplicável à gestão de conhecimentos tradicionais quando adaptado ao contexto comunitário.

A importância do conhecimento tradicional para o desenvolvimento rural, incluindo campos específicos como etnobotânica, etnoecologia e agroecologia, tem sido amplamente reconhecida na gestão de recursos, na formulação de novos caminhos para a agricultura sustentável \cite{Anderson2007}, no monitoramento da conservação e na compreensão de processos ecológicos \cite{Purwanto2022}. O conceito de desenvolvimento rural transcende o mero crescimento econômico, abrangendo dimensões socioculturais, como saúde, educação, qualidade de vida e bem-estar \cite{Bokic2014}.

No Brasil, a lei que mais se aproxima da regulamentação do direito à propriedade intelectual no rol dos sui generis é a Lei n.º 13.123/2015, que regulamenta o acesso ao patrimônio genético e a proteção ao conhecimento tradicional associado a este patrimônio genético \cite{Brasil2015}. Porém, este marco legal deixa aberturas quanto à proteção de saberes e fazeres que não sejam oriundos ou resultem em patrimônio genético. Essa lacuna sinaliza a necessidade de instrumentos legais adicionais ou de regulamentação complementar que garanta o registro e a salvaguarda de tais conhecimentos, essenciais para a conservação e o desenvolvimento sustentável.

\subsection{Gestão Territorial e Inovação Social Agroecológica}

O Sistema Nacional de Unidades de Conservação (SNUC), instituído pela Lei n.º 9.985/2000 \cite{Brasil2000}, estabelece diretrizes para gestão ambiental em territórios que abrigam comunidades tradicionais. A gestão de processos em áreas protegidas demanda instrumentos de governança que conciliem conservação da biodiversidade com valorização do conhecimento tradicional através de sistemas agroecológicos, ecoturismo responsável e aplicação sustentável de conhecimentos tradicionais.

A compreensão de processos colaborativos baseados em conhecimentos tradicionais exige articulação com \textbf{inovação social}, caracterizada pela alta intensidade de engajamento comunitário, exploração de oportunidades derivadas de conhecimento local especializado e navegação em ambientes de participação coletiva. O desenvolvimento de modelos participativos operacionaliza esse processo em fases estruturadas (identificação de ativos comunitários, validação técnica colaborativa, documentação e salvaguarda), aplicáveis ao contexto de comunidades que buscam registrar, valorar e proteger saberes tradicionais.

\subsection{Brand Equity e Valorização de Produtos Tradicionais}

A gestão estratégica de marcas coletivas emerge como instrumento de apropriação de valor econômico para conhecimentos tradicionais materializados em produtos e serviços. O conceito de \textbf{brand equity}, conforme Keller \cite{Keller1993}, refere-se ao valor agregado conferido a produtos e serviços pelo nome, símbolo ou identidade da marca, manifestando-se em percepções, preferências e comportamentos dos consumidores que transcendem atributos funcionais objetivos dos produtos.

Aaker \cite{Aaker1991} estruturou o brand equity em cinco dimensões analíticas: (i) consciência de marca (brand awareness), capacidade de identificação e recordação espontânea; (ii) associações de marca (brand associations), conjunto de atributos, benefícios e significados vinculados à marca; (iii) qualidade percebida (perceived quality), julgamento subjetivo sobre superioridade do produto; (iv) lealdade de marca (brand loyalty), compromisso de recompra e resistência à substituição; e (v) outros ativos proprietários, incluindo patentes, marcas registradas e relacionamentos em canais de distribuição. Essa estrutura conceitual, originalmente desenvolvida para contextos empresariais convencionais, pode ser adaptada para valoração de conhecimentos tradicionais quando estes se materializam em produtos diferenciados com identidade territorial.

No contexto de comunidades quilombolas, o \textbf{brand equity territorial} \cite{Anholt2007} representa ativo intangível estratégico construído através da interseção entre identidade cultural, qualidade intrínseca dos produtos e reputação coletiva. Vandecandelaere et al. \cite{Vandecandelaere2009} demonstram que produtos vinculados à origem geográfica podem capturar valor premium quando há: (i) especificidade territorial demonstrável (terroir); (ii) know-how tradicional documentado; (iii) reputação histórica comprovável; (iv) qualidade distintiva mensurável; e (v) governança coletiva institucionalizada.

A construção de brand equity para produtos quilombolas demanda instrumentos específicos de propriedade intelectual coletiva. As \textbf{Indicações Geográficas (IG)}, regulamentadas pela Lei 9.279/96 \cite{brasil1996lei9279}, constituem mecanismo de proteção que vincula produto a território, reconhecendo tanto Indicação de Procedência (IP), quando área geográfica se torna conhecida como centro de produção, quanto Denominação de Origem (DO), quando qualidade ou características do produto são atribuíveis ao meio geográfico. No Brasil, casos como Queijo Canastra (MG), Cachaça de Paraty (RJ) e Café do Cerrado Mineiro demonstram que IGs podem gerar prêmios de preço de 30\% a 150\% \cite{Bruch2009}, evidenciando potencial econômico para comunidades tradicionais.

As \textbf{marcas coletivas}, também previstas na Lei 9.279/96 \cite{brasil1996lei9279}, diferenciam-se por identificar produtos ou serviços provenientes de membros de uma entidade coletiva (associação, cooperativa), sem necessariamente vincular-se a território específico. Para comunidades quilombolas, marcas coletivas oferecem flexibilidade para: (i) proteger diversidade de produtos sob identidade comum; (ii) estabelecer padrões de qualidade administrados coletivamente; (iii) fortalecer posição negociadora em cadeias produtivas; e (iv) apropriar-se de valor agregado gerado pela reputação coletiva.

Kapferer \cite{Kapferer2008} propõe o \textbf{Prisma de Identidade de Marca}, framework que estrutura identidade de marca em seis facetas interconectadas: (i) físico (características tangíveis do produto); (ii) personalidade (caráter humano atribuído à marca); (iii) cultura (sistema de valores e princípios); (iv) relacionamento (modo de interação com consumidores); (v) reflexo (imagem do usuário típico); e (vi) autoimagem (como o consumidor se vê ao usar o produto). Aplicado a produtos quilombolas, esse prisma revela que brand equity transcende atributos funcionais, incorporando dimensões simbólicas de autenticidade, sustentabilidade, justiça social e preservação cultural que ressonam com consumidores conscientes em mercados diferenciados.

A mensuração de brand equity demanda metodologias que capturem tanto valor econômico quanto valor relacional. Keller e Lehmann \cite{KellerLehmann2006} desenvolveram a \textbf{Brand Value Chain}, modelo que conecta investimentos em marketing a valor financeiro através de três estágios: (i) programa de marketing (mix de comunicação, produto, distribuição, preço); (ii) mentalidade do cliente (consciência, associações, atitudes, apego, atividade); e (iii) desempenho de mercado (preço premium, elasticidade de preço, expansão de mercado, estrutura de custos). Para comunidades quilombolas, essa cadeia pode ser adaptada incluindo investimentos em: capacitação, certificação, desenvolvimento de embalagem, narrativa de marca (storytelling), acesso a mercados diferenciados e comunicação de atributos culturais e ambientais.

O conceito de \textbf{place branding} \cite{Anholt2007} oferece perspectiva complementar, reconhecendo que territórios podem ser geridos como marcas, construindo reputação que beneficia simultaneamente produtos, turismo, investimentos e orgulho comunitário. Para comunidades quilombolas do semiárido sergipano, estratégias de place branding podem posicionar a região como centro de referência em agroecologia tradicional, gerando spillover effects que valorizam portfólio completo de produtos e serviços territoriais. Essa abordagem alinha-se ao modelo de \textbf{marca guarda-chuva} (umbrella brand), onde identidade coletiva fortalece produtos individuais através de associações compartilhadas de qualidade, autenticidade e valores.
%%%%% AJUSTAR ESSES (III)

A literatura sobre comércio justo (fair trade) e mercados diferenciados \cite{Raynolds2007} demonstra que consumidores em mercados conscientes apresentam disposição a pagar premium (willingness to pay) por produtos que incorporam atributos de: (i) justiça social (remuneração adequada a produtores); (ii) sustentabilidade ambiental (produção agroecológica); (iii) autenticidade cultural (preservação de tradições); e (iv) transparência (rastreabilidade e governança participativa). Estudos de análise conjunta (conjoint analysis) indicam que esses atributos intangíveis podem contribuir com 25\% a 40\% do valor percebido total \cite{Loureiro2002}, evidenciando relevância econômica da gestão de brand equity para comunidades tradicionais.

A construção sistemática de brand equity territorial para produtos quilombolas demanda: (i) identificação de produtos com potencial diferenciador; (ii) documentação de especificidades territoriais e know-how tradicional; (iii) desenvolvimento de identidade visual e narrativa de marca culturalmente autêntica; (iv) estabelecimento de padrões de qualidade e protocolos de certificação; (v) registro de marcas coletivas ou indicações geográficas; (vi) capacitação comunitária em gestão de marca e marketing; (vii) acesso a canais de distribuição para mercados diferenciados; e (viii) comunicação estratégica de atributos distintivos. Esse processo, quando mediado por frameworks participativos que assegurem protagonismo comunitário, representa instrumento de inovação social que fortalece apropriação de valor econômico, autoestima coletiva e reprodução social de comunidades tradicionais.

\section{Gestão da Inovação Tecnológica e Valoração de Ativos Intangíveis}

A salvaguarda dos saberes dos Povos e Comunidades Tradicionais (PCTs) representa desafio estratégico de gestão da inovação, pois essas comunidades acumularam ao longo de séculos ativos intangíveis de alto valor cultural, ambiental e científico \cite{Gafner-Rojas2020}. A gestão desses conhecimentos como propriedade intelectual coletiva fortalece o papel dos PCTs como agentes de desenvolvimento territorial sustentável e conservação ambiental \cite{Ens2016}, gerando oportunidades para pesquisa colaborativa, interação universidade-comunidade e fortalecimento institucional comunitário.

A compreensão dos conhecimentos tradicionais inseridos em \textbf{ecossistemas de inovação} \cite{Adner2006} revela dinâmicas de interdependência tecnológica que estruturam vantagens competitivas territoriais. A análise de assimetrias upstream (componentes tecnológicos integrados) e downstream (complementos que agregam valor ao produto final) \cite{AdnerKapoor2010} ilumina estratégias para posicionamento de conhecimentos tradicionais em cadeias produtivas sustentáveis. Comunidades que dominam saberes sobre manejo de recursos naturais, por exemplo, controlam componentes upstream essenciais para bioeconomia, podendo fortalecer sua posição negociadora em arranjos de comercialização.

Os conhecimentos etnoecológicos, etnopedológicos e etnoclimatológicos acumulados \cite{Fajardo2021} configuram-se como tecnologias sociais passíveis de valoração, documentação e aplicação responsável. Tais ativos intangíveis, quando adequadamente geridos, fortalecem a resiliência organizacional das comunidades frente a mudanças ambientais e pressões econômicas \cite{Degryse2009}. A gestão estratégica desses ativos demanda articulação entre \textbf{capacidades dinâmicas} \cite{Teece2007}, que permitem sensing (identificar oportunidades), seizing (mobilizar recursos) e reconfiguring (adaptar estruturas organizacionais) em resposta a mudanças ambientais e sociais.



\subsection{Relevância para Fortalecimento de Capacidades}


Uma dimensão crítica e frequentemente negligenciada do fortalecimento de capacidades para gestão de conhecimentos tradicionais reside na sua função estratégica para garantia de direitos territoriais e reconhecimento jurídico das comunidades tradicionais. Comunidades quilombolas e outros povos tradicionais enfrentam sistematicamente desafios em processos de regularização fundiária, titulação de terras e reconhecimento legal de ocupação ancestral, frequentemente pela ausência de instrumentos formais e evidências documentais que comprovem vínculos históricos com seus territórios. 

No Brasil, dados oficiais do Instituto Nacional de Colonização e Reforma Agrária (INCRA) indicam que, das mais de 3.300 comunidades quilombolas certificadas pela Fundação Cultural Palmares, apenas cerca de 4\% possuem título definitivo de propriedade da terra \cite{INCRA2023, FCP2024}. Similarmente, segundo a Fundação Nacional dos Povos Indígenas (FUNAI), aproximadamente 30\% das terras indígenas no Brasil ainda aguardam conclusão dos processos de demarcação e homologação \cite{FUNAI2024}, evidenciando a magnitude do problema de insegurança jurídica territorial enfrentado por povos e comunidades tradicionais. Esses processos burocráticos prolongados, que podem estender-se por décadas, deixam comunidades vulneráveis a invasões, grilagem, conflitos fundiários e perda progressiva de seus territórios ancestrais \cite{Treccani2006, LeiteRDS2015}. 

O mapeamento sistemático e a documentação rigorosa de conhecimentos agroecológicos específicos,práticas de manejo adaptadas a características pedológicas, climáticas e ecológicas locais; variedades crioulas desenvolvidas ao longo de gerações; sistemas de rotação e consórcio culturalmente situados, constituem evidências materiais robustas de ocupação tradicional prolongada e uso sustentável do território. Esses registros técnico-científicos podem funcionar como instrumentos probatórios em processos administrativos e judiciais de reconhecimento territorial, fortalecendo a posição jurídica das comunidades frente a contestações, invasões e disputas fundiárias. Adicionalmente, a formalização de conhecimentos tradicionais por meio de protocolos de registro participativo, associada a mecanismos de propriedade intelectual coletiva (indicações geográficas, marcas coletivas, registros sui generis), gera documentação oficial que atesta não apenas a existência de saberes especializados, mas sua vinculação indissociável ao território de origem, reforçando argumentos jurídicos para garantia de direitos territoriais e autodeterminação comunitária.

Do ponto de vista do marco legal brasileiro, a Constituição Federal de 1988 representa avanço fundamental nos direitos dos povos tradicionais, garantindo no artigo 68 do Ato das Disposições Constitucionais Transitórias (ADCT) que \textit{aos remanescentes das comunidades dos quilombos que estejam ocupando suas terras é reconhecida a propriedade definitiva, devendo o Estado emitir-lhes os títulos respectivos} \cite{Brasil1988}. A Convenção 169 da Organização Internacional do Trabalho (OIT), ratificada pelo Brasil através do Decreto nº 5.051/2004 e posteriormente promulgada pelo Decreto nº 10.088/2019, reconhece a autoidentificação como critério fundamental para definição de povos e comunidades tradicionais, estabelecendo direitos territoriais e culturais específicos \cite{OIT2004}. Entretanto, a ausência de legislação específica para regularização territorial de diversas categorias de povos tradicionais, faxinalenses, ribeirinhos, pescadores artesanais, quebradeiras de coco-babaçu, entre outros, representa lacuna grave no ordenamento jurídico brasileiro. 

Conforme documento protocolado por mais de 600 entidades representativas de povos e comunidades tradicionais, \textit{não há órgão com a competência institucional para processar a demanda territorial de povos tradicionais} que não sejam indígenas ou quilombolas \cite{Thum2017}. Poucos estados desenvolveram legislações próprias para preencher essa lacuna, destacando-se o Paraná com a Lei do Faxinal (Lei Estadual nº 15.673/2007), o Maranhão com a Lei nº 11.399/2020 do Estatuto Estadual da Igualdade Racial, e Goiás com a Lei nº 21.013/2021 da Política Estadual de Desenvolvimento Sustentável dos Povos e Comunidades Tradicionais.

Os dados e evidências apresentados confirmam que a gestão de conhecimentos tradicionais assume função estratégica para garantia de direitos territoriais. A capacitação técnica das comunidades para documentação sistemática de vínculos ancestrais, história oral, práticas culturais e padrões de ocupação tradicional é fundamental para superação de barreiras burocráticas e judiciais que impedem o reconhecimento territorial. Apesar dos avanços constitucionais e da ratificação de instrumentos internacionais, o ritmo de regularização fundiária permanece dramaticamente insuficiente diante das demandas existentes. A ausência de instrumentos formais e evidências documentais padronizadas continua sendo argumento recorrentemente utilizado em instâncias administrativas e judiciais para negar direitos constitucionalmente garantidos, perpetuando injustiças históricas e vulnerabilizando comunidades à violência, grilagem e perda territorial. Nesse contexto, o fortalecimento de capacidades comunitárias para gestão, documentação e sistematização de conhecimentos tradicionais, aliado ao desenvolvimento de protocolos participativos de registro e valoração, emerge como estratégia indispensável para efetivação da justiça territorial no Brasil e para garantia da reprodução social, cultural e econômica de povos e comunidades tradicionais em seus territórios ancestrais.

A relevância para o fortalecimento de capacidades reside na proposição de frameworks inovadores de empoderamento comunitário para proteção de conhecimentos tradicionais como ativos estratégicos, desenvolvimento de protocolos participativos de fortalecimento organizacional para gestão de propriedade intelectual coletiva, e estabelecimento de programas de capacitação local para proteção técnica e legal. O projeto contribui para consolidação de capacidades territoriais de inovação social que integram universidades, comunidades tradicionais e instituições de pesquisa, alinhando-se ao modelo da Tríplice Hélice em sua aplicação a contextos de desenvolvimento territorial sustentável com foco em agência comunitária e soberania epistemológica.

\section{Problema de Pesquisa}
\subsection{Fortalecimento de Capacidades para Proteção de Conhecimentos Agroecológicos}

Apesar do reconhecimento estratégico de conhecimentos tradicionais como ativos intangíveis de alto valor para a ciência e a sustentabilidade \cite{Santos2020}, persiste lacuna crítica em capacidades organizacionais comunitárias para a valoração, proteção e aplicação responsável desses saberes. As metodologias convencionais de gestão de propriedade intelectual, desenvolvidas para contextos empresariais tradicionais, não respondem adequadamente às necessidades de comunidades tradicionais que carecem de capacidade absortiva para reconhecer e integrar mecanismos de proteção de propriedade intelectual, capacidades de valoração comunitária participativa, capacidades dinâmicas para adaptação e inovação, bem como marcos de empoderamento que conciliem pesquisa científica com preservação e soberania cultural \cite{ANA2007, Embrapa2014}.

No contexto das comunidades quilombolas do semiárido baiano, observa-se erosão acelerada de capacidades organizacionais tradicionais devido à ausência de modelos de fortalecimento para gestão de propriedade intelectual coletiva, à falta de metodologias participativas para valoração de conhecimentos \cite{Albagli2021, GuiaMapa2022}, à descontinuidade na transmissão intergeracional de capacidades técnicas e ao desaparecimento progressivo de variedades crioulas e sistemas de manejo tradicionalmente conservados \cite{Embrapa2021, Oliveira2020}. Adicionalmente, observam-se limitações em marcos de empoderamento que fortaleçam a agência comunitária. Os marcos regulatórios existentes, particularmente a Lei nº 13.123/2015 \cite{brasil2015lei13123} e o Sistema Nacional de Unidades de Conservação (SNUC), revelam-se insuficientes para suportar o fortalecimento de capacidades organizacionais comunitárias baseadas em conhecimentos tradicionais.

Nessa perspectiva, a dimensão crítica e frequentemente negligenciada do fortalecimento de capacidades para gestão de conhecimentos tradicionais reside na sua função estratégica para garantia de direitos territoriais e reconhecimento jurídico de comunidades tradicionais \cite{Thum2017}. A Constituição Federal de 1988, em seu artigo 68 do Ato das Disposições Constitucionais Transitórias (ADCT), garantiu aos remanescentes de quilombos o direito à propriedade definitiva de suas terras \cite{Brasil1988}, marco legal posteriormente complementado pela ratificação da Convenção 169 da OIT, que reconhece a autoidentificação como critério fundamental para definição de povos e comunidades tradicionais \cite{OIT2004}. Entretanto, a efetivação desses direitos constitucionais enfrenta obstáculos estruturais de magnitude alarmante.

Dados oficiais do Instituto Nacional de Colonização e Reforma Agrária (INCRA) revelam que apenas 273 títulos foram emitidos desde 1988, regularizando somente 1.049.283 hectares (0,11\% do território brasileiro) para 323 comunidades quilombolas \cite{INCRA2023}. Das mais de 3.300 comunidades certificadas pela Fundação Cultural Palmares, apenas 4,3\% da população quilombola vive em terras tituladas \cite{FCP2024}. 

Projeções indicam que, mantido o ritmo atual, seriam necessários 2.188 anos para titular integralmente os 1.802 processos abertos no INCRA, evidenciando a inviabilidade prática do modelo vigente de regularização fundiária \cite{Treccani2006}. Situação análoga ocorre com povos indígenas: aproximadamente 30\% das terras indígenas ainda aguardam conclusão dos processos de demarcação e homologação \cite{FUNAI2024}.

Um dos obstáculos centrais ao reconhecimento territorial reside na exigência de comprovação documental de vínculos históricos e ocupação ancestral. Comunidades quilombolas frequentemente carecem dessa documentação formal, apesar de manterem vínculos ancestrais comprovados por outras fontes como memória oral, práticas culturais e conhecimentos agroecológicos territorialmente situados \cite{LeiteRDS2015}. O Relatório Técnico de Identificação e Delimitação (RTID), instrumento central no processo de regularização quilombola, deve conter dados antropológicos, levantamento fundiário e memorial descritivo da comunidade; entretanto, dos processos abertos no INCRA, somente 17\% avançaram até a etapa de publicação do RTID, evidenciando gargalos estruturais no sistema \cite{Treccani2006}.

A falta de titulação territorial gera consequências graves para as comunidades. Pesquisa da Coordenação Nacional de Articulação das Comunidades Negras Rurais Quilombolas (CONAQ) identificou que 65\% dos assassinatos de quilombolas entre 2018 e 2022 aconteceram em territórios não titulados, sendo 70\% desses crimes motivados por conflitos fundiários \cite{conaq2023racismo}. A ausência de segurança jurídica impede o acesso a políticas públicas específicas em áreas como educação, saúde, moradia e preservação cultural, perpetuando injustiças históricas e vulnerabilizando comunidades à violência, grilagem e desmatamento.

Nesse contexto, a documentação sistemática e a valoração técnico-científica de conhecimentos tradicionais agroecológicos assumem função estratégica. O mapeamento rigoroso de práticas de manejo adaptadas a características pedológicas, climáticas e ecológicas locais \cite{Santos2019, Sampaio2022}; a identificação de variedades crioulas desenvolvidas ao longo de gerações \cite{Embrapa2021, Oliveira2020}; a sistematização de sistemas de rotação, consórcio e quintais produtivos culturalmente situados \cite{Ramos2021, Scielo2009}; e o reconhecimento de conhecimentos etnoecológicos sobre manejo de recursos hídricos e energéticos em contextos de agricultura familiar \cite{Gomes2024} constituem evidências materiais robustas de ocupação tradicional prolongada e uso sustentável do território. Esses registros técnico-científicos, obtidos por meio de metodologias participativas de mapeamento territorial \cite{GuiaMapa2022, Albagli2021}, podem funcionar como instrumentos probatórios em processos administrativos e judiciais de reconhecimento territorial, fortalecendo a posição jurídica das comunidades frente a contestações, invasões e disputas fundiárias.


\subsection{Centralidade de Pesquisa}

Diante desse cenário, o problema central desta pesquisa consiste em desenvolver um modelo de gestão estratégica de ativos intangíveis baseado em valoração por machine learning que integre tecnologias computacionais avançadas ao fortalecimento de capacidades comunitárias para proteção de sistemas agrícolas tradicionais quilombolas. Esse modelo deve combinar processamento de linguagem natural para mapeamento automatizado de conhecimentos agroecológicos como ativos intangíveis estratégicos; metodologias participativas de valoração que reconheçam simultaneamente dimensões culturais, científicas e territoriais desses saberes; frameworks de capacitação para proteção de propriedade intelectual coletiva aliados à produção de evidências técnicas para processos de regularização fundiária; protocolos de fortalecimento organizacional que assegurem gestão colaborativa e autonomia comunitária; e marcos éticos de pesquisa participativa que garantam respeito aos direitos, agência comunitária e sustentabilidade territorial.

A pesquisa busca responder ao seguinte questionamento: como um modelo de gestão estratégica de ativos intangíveis baseado em machine learning pode fortalecer capacidades comunitárias de documentação, valoração e proteção de conhecimentos tradicionais agroecológicos em comunidades quilombolas do semiárido brasileiro, contribuindo simultaneamente para salvaguarda cultural, aplicabilidade científica e garantia de direitos territoriais? O desafio central reside em desenvolver "framework" metodológico computacionalmente assistido que integre rigor técnico-científico com legitimidade epistemológica comunitária, reconhecendo conhecimentos tradicionais não apenas como patrimônio cultural a ser preservado, mas como conjunto de capacidades estratégicas para inovação social, desenvolvimento territorial sustentável e fortalecimento jurídico-institucional de comunidades quilombolas, operacionalizando essa visão por meio de algoritmos de machine learning culturalmente situados e validados participativamente.


\section{Questões de Pesquisa}

\begin{itemize}

\item \textbf{Q1 – Capacidade Absortiva e Propriedade Intelectual:} 
Como técnicas de \textit{machine learning} podem fortalecer a capacidade absortiva organizacional das comunidades na identificação e valoração de seus ativos intangíveis tradicionais tais como sistemas agroecológicos tradicionais?

\item \textbf{Q2 – Valoração Participativa de Conhecimentos Tradicionais:} 
De que forma a integração entre instrumentos psicométricos, modelos econômicos de valoração e validação comunitária pode aumentar a capacidade das comunidades em mensurar o valor estratégico de seus conhecimentos tradicionais?

\item \textbf{Q3 – Capacidades Dinâmicas para Inovação Social:} 
Como o desenvolvimento de capacidades dinâmicas (sensing, seizing, reconfiguring) pode favorecer a conversão de conhecimentos tradicionais em inovação social comunitária?

\item \textbf{Q4 – Equilíbrio Multiobjetivo em Decisões Comunitárias:} 
Em que medida algoritmos de otimização multiobjetivo podem fortalecer a capacidade comunitária de equilibrar preservação cultural e aplicabilidade científica dos conhecimentos tradicionais?

\item \textbf{Q5 – Capacitação para Proteção de Propriedade Intelectual em Sistemas Agroecológicos:} 
Quais mecanismos de proteção de propriedade intelectual (indicações geográficas, marcas coletivas, proteções sui generis) são mais efetivos para capacitar comunidades quilombolas a garantir apropriação justa de valor econômico em sistemas agroecológicos tradicionais?

\item \textbf{Q6 – Fortalecimento de Ecossistemas Colaborativos de Inovação:} 
De que forma o fortalecimento da capacidade comunitária de integração em ecossistemas colaborativos de inovação, mediado por marcos de gestão de propriedade intelectual, facilita a conversão de práticas agroecológicas em tecnologias sociais escaláveis?

\item \textbf{Q7 – Brand Equity e Apropriação de Valor em Produtos Tradicionais:} 
Como a construção de brand equity territorial, através de marcas coletivas e indicações geográficas, pode fortalecer a apropriação de valor econômico por comunidades quilombolas em cadeias produtivas de produtos agroecológicos tradicionais, considerando dimensões de consciência de marca, associações de qualidade, lealdade e prêmio de preço?

\end{itemize}


\section{Hipóteses}

\begin{itemize}

\item \textbf{H1 – Capacidade Absortiva e Propriedade Intelectual:} 
O fortalecimento da capacidade absortiva organizacional através de frameworks baseados em \textit{machine learning} aumenta a eficiência comunitária na identificação e valoração de ativos intangíveis tradicionais.

\item \textbf{H2 – Valoração Participativa de Conhecimentos Tradicionais:} 
A integração entre instrumentos psicométricos, modelos econômicos de valoração e validação comunitária fortalece a capacidade das comunidades em mensurar consistentemente o valor estratégico de seus conhecimentos tradicionais.

\item \textbf{H3 – Capacidades Dinâmicas para Inovação Social:} 
O desenvolvimento de capacidades dinâmicas (sensing, seizing, reconfiguring) nas comunidades favorece a conversão de conhecimentos tradicionais em inovação social comunitária sustentável.

\item \textbf{H4 – Equilíbrio Multiobjetivo em Decisões Comunitárias:} 
Algoritmos de otimização multiobjetivo fortalecem a capacidade comunitária de equilibrar preservação cultural e aplicabilidade científica em contextos de gestão de conhecimentos tradicionais.

\item \textbf{H5 – Capacitação para Proteção de Propriedade Intelectual em Sistemas Agroecológicos:} 
Mecanismos de proteção de propriedade intelectual adaptados aos sistemas agroecológicos tradicionais, integrados em programas de capacitação comunitária, potencializam a apropriação justa de valor econômico pelas comunidades quilombolas e fortalecem sua posição em cadeias produtivas sustentáveis.

\item \textbf{H6 – Fortalecimento de Ecossistemas Colaborativos de Inovação:} 
O fortalecimento da capacidade comunitária de integração em ecossistemas colaborativos de inovação, mediado por frameworks de gestão de propriedade intelectual, acelera a conversão de práticas agroecológicas em tecnologias sociais protegidas com viabilidade de escala territorial.
%%%%AJUSTAR
\item \textbf{H7 – Brand Equity e Geração de Valor Econômico:} 
A adoção de estratégias de branding coletivo por meio de marcas coletivas, indicações geográficas e certificações, mediada por narrativas de autenticidade cultural e sustentabilidade socioambiental, aumenta o valor percebido (brand equity) dos produtos tradicionais quilombolas em comparação a produtos similares sem tais estratégias.

\end{itemize}


\section{Objetivos}
\subsection{Objetivo Geral}

Desenvolver um modelo de gestão estratégica de ativos intangíveis baseado em valoração por machine learning para a proteção de sistemas agrícolas tradicionais quilombolas, integrando frameworks de propriedade intelectual coletiva com metodologias participativas de fortalecimento de capacidades comunitárias.

\subsection{Objetivos Específicos}

\begin{itemize}

\item Compreender a fundamentação teórica sobre de propriedade intelectual e gestão de conhecimentos tradicionais, analisando marcos legais, instrumentos de proteção e mecanismos de registro participativo que assegurem apropriação ética e reconhecimento coletivo dos saberes quilombolas.

\item Identificar e analisar o potencial dos sistemas agroecológicos tradicionais como ativos intangíveis, mapeando dimensões de valor cultural, científico e econômico que orientem sua valoração comunitária e incorporação em estratégias de gestão do conhecimento.

\item Desenvolver e validar metodologias de valoração participativa adaptadas, integrando ferramentas de \textit{machine learning} e frameworks econômicos (VAIC, TRI) com validação comunitária para mensuração e classificação de ativos intangíveis estratégicos.

\item Capacitar comunidades na identificação e recomendação de mecanismos apropriados de proteção de propriedade intelectual para sistemas agroecológicos tradicionais (indicações geográficas, marcas coletivas, proteções sui generis), integrando estratégias de construção de brand equity territorial que assegurem apropriação justa de valor econômico, fortalecimento de posição negociadora em cadeias produtivas sustentáveis e desenvolvimento de identidade coletiva diferenciada em mercados conscientes.

\item Estruturar modelos participativos de ecossistemas de inovação colaborativa que integrem conhecimentos agroecológicos tradicionais com pesquisa científica, fortalecendo a capacidade comunitária de conversão de saberes em tecnologias sociais escaláveis e sustentáveis.

\item Criar, validar e operacionalizar um modelo integrado de fortalecimento de capacidades para proteção e valorização de conhecimentos tradicionais, combinando processamento automatizado, integração multidimensional de dados e validação participativa com as comunidades quilombolas do semiárido baiano.


\end{itemize}


\section{Justificativa}

A crescente preocupação global com a conservação da biodiversidade e o desenvolvimento sustentável tem evidenciado a importância fundamental dos conhecimentos tradicionais para a preservação dos ecossistemas e a promoção de práticas agrícolas sustentáveis. No contexto brasileiro, as comunidades quilombolas representam um patrimônio cultural e ecológico de valor inestimável, detendo saberes agroecológicos acumulados ao longo de séculos de interação harmoniosa com seus territórios, configurando sistemas adaptativos complexos de alta relevância socioecológica.

Esses conhecimentos tradicionais, desenvolvidos através de observação sistemática e experimentação empírica, constituem repositórios estruturados de informações sobre manejo de recursos naturais que demonstram eficácia comprovada na conservação da biodiversidade e na adaptação às variabilidades climáticas do semiárido. Entretanto, esses saberes enfrentam ameaças crescentes devido a processos de urbanização, mudanças nos padrões de uso da terra, êxodo rural e descontinuidade na transmissão intergeracional do conhecimento, demandando urgentemente estratégias de mapeamento, registro e salvaguarda tecnologicamente avançadas.

A necessidade de desenvolvimento e aplicação de metodologias computacionalmente para o mapeamento, registro e validação desses conhecimentos torna-se cada vez mais urgente. A convergência de tecnologias emergentes de inteligência artificial, incluindo algoritmos de processamento de linguagem natural (BERT, RoBERTa), modelos ensemble de machine learning (Random Forest, Gradient Boosting, SVM), redes neurais convolucionais (CNN) e algoritmos de otimização multiobjetivo (NSGA-II, MOEA/D), oferece oportunidades inéditas para sistematizar e preservar esses saberes com precisão quantitativa e reprodutibilidade metodológica, superando limitações tradicionais dos instrumentos etnográficos convencionais em termos de escala temporal, acurácia preditiva e validação estatística.

Esta pesquisa justifica-se pela contribuição metodológica que oferece ao integrar técnicas avançadas de inteligência artificial com princípios de pesquisa participativa, criando um framework híbrido que preserva a legitimidade epistemológica dos conhecimentos tradicionais enquanto aplica rigor técnico-científico ao seu mapeamento e validação. A abordagem proposta preenche uma lacuna metodológica significativa na área de estudos etnográficos aplicados à conservação, estabelecendo protocolos de validação psicométrica culturalmente adaptados com indicadores estatísticos específicos ($\alpha$ de Cronbach $\geq$ 0,80; CFI $\geq$ 0,95; RMSEA $\leq$ 0,06) e métricas de performance computacional mensuráveis (acurácia $\geq$ 85\% para NLP; AUC-ROC $\geq$ 0,90 para modelos ensemble; F1-score $\geq$ 0,85 para categorização automática).

Do ponto de vista técnico-científico, o estudo contribui para o avanço do campo da etnografia digital e da aplicação de machine learning em ciências sociais, desenvolvendo protocolos específicos para trabalho com povos e comunidades tradicionais que respeitem princípios éticos de consentimento livre, prévio e informado. A pesquisa estabelece precedentes metodológicos para futuras investigações na área através da implementação de: (i) algoritmos de processamento de linguagem natural (BERT/RoBERTa) para categorização automática de práticas agroecológicas; (ii) modelos psicométricos baseados na Teoria de Resposta ao Item (TRI) para mensuração precisa do conhecimento tradicional; (iii) análise fatorial confirmatória para identificação de estruturas dimensionais latentes nos saberes quilombolas; (iv) redes neurais convolucionais para reconhecimento de padrões textuais; e (v) algoritmos de otimização multiobjetivo para balanceamento entre preservação cultural e aplicabilidade científica, consolidando um framework computacional inédito para estudos etnográficos quantitativos.

Do ponto de vista da conservação ambiental, a pesquisa contribui para a preservação da biodiversidade do semiárido brasileiro, bioma caracterizado por alta diversidade biológica e endemismo, mas que enfrenta pressões crescentes devido às mudanças climáticas e atividades antrópicas. O reconhecimento e a valorização computacionalmente assistida dos conhecimentos tradicionais podem fornecer estratégias de adaptação e manejo sustentável fundamentais para a resiliência desses ecossistemas. A aplicação de modelos ensemble de machine learning possibilitará a identificação de práticas agroecológicas com maior potencial de escalabilidade e replicabilidade em sistemas produtivos sustentáveis, enquanto a validação participativa garantirá a manutenção da autenticidade cultural e da soberania epistemológica das comunidades quilombolas. Adicionalmente, os algoritmos de otimização multiobjetivo desenvolvidos permitirão equilibrar objetivos aparentemente conflitantes entre conservação cultural e aplicabilidade científica, gerando soluções Pareto-ótimas que beneficiem simultaneamente a preservação dos saberes tradicionais e sua contribuição para estratégias contemporâneas de adaptação climática e conservação da biodiversidade.

\section{Delineamento do estudo}

Este estudo configura-se como uma pesquisa exploratória de métodos mistos com abordagem tecnológica avançada, combinando machine learning, psicometria e metodologias participativas em um framework computacional híbrido. O delineamento metodológico está estruturado em seis fases sequenciais e integradas:

\begin{enumerate}[label=\bfseries(\Roman*)] 
  \item \textbf{Revisão sistemática:} Mapeamento sistemático da literatura sobre aplicações de algoritmos de machine learning (ensemble methods, CNN, NLP) em estudos etnográficos, incluindo análise bibliométrica e identificação de gaps metodológicos em etnografia digital aplicada a conhecimentos tradicionais agroecológicos;
  
  \item \textbf{Desenvolvimento e validação psicométrica de instrumentos:} Construção de instrumentos de coleta culturalmente adaptados para comunidades quilombolas, seguida de validação mediante Teoria de Resposta ao Item (TRI), análise fatorial confirmatória (CFA) e testes de invariância métrica, visando atingir indicadores de confiabilidade ($\alpha$ $\geq$ 0,80) e ajuste estrutural (CFI $\geq$ 0,95; RMSEA $\leq$ 0,06);
  
  \item \textbf{Implementação e benchmarking de algoritmos ML:} Desenvolvimento e teste comparativo de algoritmos de machine learning (Random Forest, Gradient Boosting, SVM, CNN) para categorização automática de conhecimentos agroecológicos, incluindo processamento de linguagem natural (BERT/RoBERTa) para análise textual, com avaliação de performance através de métricas específicas (acurácia $\geq$ 85\%, AUC-ROC $\geq$ 0,90, F1-score $\geq$ 0,85);
  
  \item \textbf{Modelagem estatística multivariada e análise de correspondência:} Aplicação de técnicas estatísticas avançadas para investigação de associações entre variáveis sociodemográficas e domínios de conhecimento agroecológico, incluindo análise de correspondência múltipla, modelagem de equações estruturais e identificação de perfis de especialização nas comunidades através de análise de clusters;
  
  \item \textbf{Validação participativa e co-criação metodológica:} Processo iterativo de validação comunitária dos resultados computacionais através de oficinas participativas, grupos focais e reuniões comunitárias, respeitando protocolos éticos de consentimento livre, prévio e informado, com integração de feedback qualitativo na calibração dos modelos quantitativos;
  
  \item \textbf{Otimização multiobjetivo e consolidação do framework:} Desenvolvimento do modelo metodológico final utilizando algoritmos de otimização multiobjetivo (NSGA-II, MOEA/D) para balanceamento entre preservação cultural e aplicabilidade científica, gerando soluções Pareto-ótimas que integrem mapeamento automatizado, validação participativa e salvaguarda tecnológica de conhecimentos agroecológicos tradicionais.
\end{enumerate}

